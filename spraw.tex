
\documentclass[12pt, a4paper, twoside]{book}
% TeX encoding = utf8
% TeX spellcheck = pl_PL 
\usepackage[OT4]{polski}
\usepackage[utf8]{inputenc} 
\usepackage{graphicx}











\author{Maciej Lotz }
\title{Robot Irp-6 w zadaniu śledzenia konturu}
\date{\today}



\begin{document}
	\sloppy
		
		
		\thispagestyle{empty}
		
		\begin{titlepage}
			\noindent
			\begin{flushright}
				\begin{tabular}[t]{r}
					\scriptsize Rok akademicki 2014/2015\\[5mm]
				\end{tabular}
			\end{flushright}
			\begin{center}
				\begin{tabular}[t]{c}
					\scriptsize POLITECHNIKA WARSZAWSKA\\
					\scriptsize WYDZIAŁ ELEKTRONIKI I~TECHNIK INFORMACYJNYCH\\
					\scriptsize INSTYTUT AUTOMATYKI I~INFORMATYKI STOSOWANEJ
				\end{tabular}
			\end{center}
			\vfill
			\begin{center}
				\resizebox{3.5cm}{!}
				{
					\includegraphics{images/logo_politechnika.pdf}
				}
				\\[5mm]\textbf{PRACOWNIA DYPLOMOWA 1}\\[3mm] \textbf{SPRAWOZDANIE} \\[15mm]
				\large Maciej Lotz\\[10mm]
				\Large \textbf{Robot \mbox{\mbox{IRp-6}} w zadaniu śledzenia konturu}
			\end{center}
			
			\vspace{20mm}
			\begin{flushright}
				\begin{tabular}{l}
					Opiekun pracy:\\
					\large dr inż. Tomasz Winiarski
				\end{tabular}
			\end{flushright}
			\vfill
			\begin{tabular}{c}
				\scriptsize Ocena pracy: \dotfill\\[10mm]
				\scriptsize \makebox[55mm]{\dotfill}\\
				\scriptsize Data i~podpis Promotora\\
				
			\end{tabular}
		\end{titlepage} 	


\newpage
\tableofcontents

\chapter{Wymagania stawiane pracy}
\chapter{Wstęp teoretyczny}
\chapter{Opis tego co zrobiono dotychczas}
	\section{Konfiguracja środowiska}
	\section{Opanowanie podstaw języka Python}
	\section{Wykonanie specjalistycznego narzędzia do śledzenia krawędzi}
	\section{Wykonanie ćwiczeń}
		\subsection{Rysowanie kwadratu w powierzu}
		\subsection{Znajdowanie środka okręgu na podstawie trzech punktów}
	\section{Zrealizowanie śledzenia prostej}
\chapter{Plany na kolejny semestr}
	\section{Zrealizowanie śledzenia konturu}
	\section{Opanowanie OpenCV i DisCODe}
\end{document}